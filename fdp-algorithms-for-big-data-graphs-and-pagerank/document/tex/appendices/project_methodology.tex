% !TEX root = ../../document.tex

\documentclass{subfiles}

\begin{document}

  \chapter{Metodología de Trabajo}
  \label{chap:methodology}

    \paragraph{}
    La metodología seguida para la realización de un proyecto de gran envergadura para un estudiante de grado, tal como es el trabajo de fin de grado requiere de una definición adecuada de la misma. De esta manera, se puede clarificar el camino a seguir para la elaboración del mismo.

    \paragraph{}
    Tal y como se puede apreciar en el documento \emph{Portal of research methods and methodologies for research projects and degree projects} \cite{haakansson2013portal}, el cual se ha utilizado como base de partida para conocer las posibles metodologías a seguir para un proyecto de investigación, esta decisión debe ser tomada al comienzo del mismo. De esta manera, se simplifica en gran medida la búsqueda de objetivos a logra durante el trabajo. En dicho documento \cite{haakansson2013portal} se realiza una diferenciación entre estrategias de investigación y métodos de investigación para después describir cada una de ellas. Posteriormente en este apéndice se indicará cómo se ha desarrollado este trabajo siguiendo dichas diferenciaciones

    \paragraph{}
    Un problema común que surge durante el desarrollo de los primeros proyectos de investigación es la dificultad por concretar el tema de estudio en un ámbito específico y analizable apropiadamente. Existen distintas razones por las cuales puede suceder dicho problema, sin embargo, la razón más destacada es la elevada interrelación entre las distintas ramas del conocimiento, que muchas veces complica la labor de \say{aislar} un pequeño sub-conjunto de ideas sobre las cuales realizar una profundización más extensa.

    \paragraph{}
    Tal y como se ha indicado anteriormente, es necesario indicar tanto las estrategias de investigación como lo métodos de investigación seguidos durante este trabajo. Sin embargo, lo primero es indicar a qué se refieren cada una de ellas. Cuando se habla de \emph{Estrategia de Investigación} nos estamos refiriendo al propósito final que se pretende conseguir con la realización del proyecto, es decir, es algo similar a los objetivos del mismo. En cambio, cuando se habla de \emph{Métodos de Investigación} nos estamos refiriendo al conjunto de \say{herramientas} conceptuales que se utilizan para conseguir llegar al propósito del trabajo.

    \paragraph{}
    La estrategia de investigación seguida a lo largo del desarrollo de este proyecto ha sido un estudio (\emph{survey}) realizado con la finalidad de tratar de comprender mejor el amplísimo área de investigación relacionado con el \emph{Big Data}, a partir del cual se ha ido profundizando en un ámbito más concreto: el estudio de grafos de tamaño masivo y la implementación del \emph{PageRank}, que en conjunto han otorgado una visión detallada acerca de dichas áreas desde una perspectiva clara. En cuanto al método seguido, lo primero fue prefijar el tema del \emph{Big Data} por el tutor del proyecto. A partir de este punto se puede diferenciar el método en dos partes, la primera de ellas correspondiente a un periodo de \emph{investigación descriptiva}, basada en obtener una visión global sobre las áreas de investigación del \emph{Big Data}, por tanto, esta fase fue guiada por las competencias de distintos cursos impartidos en universidades de a lo largo del mundo, centrados en la materia. Tras haber conseguido un nivel de comprensión adecuado del mismo, se modificó la metodología a seguir, por la que en \cite{haakansson2013portal} denominan \emph{investigación fundamental} (\emph{Fundamental Research}). Esta se caracteriza por la focalización del trabajo en un ámbito concreto a partir de la curiosidad personal, que en este caso, poco a poco fue acercandose hacia el estudio de grafos.

    \paragraph{}
    Puesto que el trabajo de fin de grado se refiere a una titulación de \emph{Ingeniería Informática}, se creyó interesante que este no se basara únicamente en tareas de investigación, sino que también contuviera una pequeña parte de implementación de código fuente. Para dicha labor, se escogió el algoritmo \emph{PageRank} puesto que a partir de su estudio se abarcan una gran cantidad de conceptos relacionados con el resto de los estudiados.

    \paragraph{}
    Muchas de estas decisiones fueron tomadas conforme avanzaba el proyecto, lo cual presenta distintas ventajas e inconvenientes. La ventaja más notoria se corresponde con el grado de libertad que se ha tenido durante todo el proceso, lo cual ha permitido centrarse en aquellas partes más motivadoras. Sin embargo, esto también genera una serie de desventajas, entre las que se encuentra la dificultad al llegar a puntos a partir de los cuales no saber hacia qué dirección continuar. Sin embargo, esta desventaja también representa un aprendizaje, que ayudará en futuras ocasiones a hacer frente a problemas semejantes con un grado de presión mucho menor debido a la experiencia adquirida en este caso.

    \paragraph{}
    Muchos proyectos de ingeniería informática se realizan siguiendo distintas metodologías de gestión de proyectos, tales como \emph{metodologías en cascada} o las denominadas \emph{ágiles} como \emph{SCRUM}. En las primeras semanas de la realización de este proyecto, se pretendió seguir una metodología basada en \emph{SCRUM}, tratando de realizar una serie de tareas divididas en bloques de dos semanas. Sin embargo, dicho enfoque se avandonó rápidamente por la naturaleza inherente de investigación seguida para este proyecto. La razón se debe a que es muy complicado compaginar las tareas de investigación, las cuales se basan en la adquisición de conocimiento con un determinado concepto concreto y un periodo de tiempo acotado.

    \paragraph{}
    Las razones que han llevado a pensar esto están relacionadas con el proceso de investigación basado en la lectura de artículos de carácter científico, los cuales se relacionan fuertemente unos con otros. Dicho suceso conlleva la necesidad de tener que leer un grupo de artículos relacionados con el que se pretende comprender mediante el conjunto de citaciones que asumen el conocimiento de los términos Extraídos de otro artículo. Estas razones dificultan la tarea de estimación temporal necesaria para entender la idea descrita en un artículo, que muchas veces se reduce a unos pocos trabajos, que además han sido comprendidos previamente, mientras que en otras ocasiones es necesario adquirir una gran cantidad de nuevos conceptos.

    \paragraph{}
    A este factor se le añade otra dificultad derivada del mismo, en un gran número de ocasiones dichas dificultades no se conocen hasta que no se ha profundizado en el entendimiento del trabajo, por lo que no se pueden estimar a priori. Sin embargo, son de gran ayuda los estudios (\emph{surveys}) realizados sobre temas específicos, que permiten obtener una visión panorámica acerca del tema mediante la lectura de un único trabajo, que después puede ser ampliada por el lector mediante la lectura de las referencias contenidas en el estudio.

    \paragraph{}
    A partir de todos estos factores se ha permitido conocer en mayor detalle cómo es el proceso de investigación, así como los retos metodológicos que surgen durante la realización de dichos proyectos. Ha habido muchos puntos que se podrían haber realizado de una manera más apropiada, con la consiguiente reducción de tiempo en dichas tareas. Sin embargo, se cree que todas estas complicaciones han permitido adquirir una experiencia que en futuras ocasiones agilizará el proceso y permitirá evitar dichos errores.


\end{document}
